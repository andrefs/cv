%!TEX TS-program = xelatex
%\documentclass[print,a4paper]{friggeri-cv}
%\documentclass[a4paper,printbw]{friggeri-cv}
%\documentclass[a4paper,print]{friggeri-cv}
%\documentclass[a4paper,web]{friggeri-cv}
\documentclass[a4paper]{friggeri-cv}
\addbibresource{aspubs.bib}

\begin{document}
\header{andré}{santos}
       {nlp, webdev and sysadmin}


% In the aside, each new line forces a line break

\begin{aside}
  \section{about}
\begin{sensitive}
  	\subsection{address}
  	Pq. Resid. Fonte Seca
	LT 8B 1oE
	4715-229 Braga
	Portugal
  	\subsection{phone}
	+351 927 715 379
  	\subsection{birth date}
    24.02.1988
\end{sensitive}
  	\subsection{website}
	\url{andrefs.com}
  	\subsection{mail}
	\href{mailto:cv@andrefs.com}{cv@andrefs.com}
  	\subsection{twitter}
	\href{http://witter.com/about_andrefs}{@about\_andrefs}
  \section{languages}
  	native portuguese
	fluent english
    basic spanish+french
  \section{programming}
    {$\varheartsuit$} Perl (Moose, Dancer, Catalyst, DBIC)
	Bash, \LaTeX
\end{aside}

\section{interests}

{\large{web development, open source software, Perl, natural language
processing, information retrieval, bioinformatics}}

\section{experience}
\begin{entrylist}
  \entry
    {{\footnotesize since} 2011}
    {GroubBuddies}
    {System administrator}
    {
	Startup company which develops web-based solutions\\
	for groups and organizations. \footnotesize{Typical tasks include:
		\begin{itemize}
			\item deployment and management of several Unix-based
			servers,\\ 
			with Apache and MySQL services
			\item managing Ruby on Rails applications on a production\\
			environment
			\item managing Git repositories
			\item development of helper scripts to automate and
			facilitate server\\
			management
			\end{itemize}
		\url{http://groupbuddies.com}
		}
	}
  \entry
    {\parbox[t][][t]{1.8cm}{{\footnotesize since}\\2012 {\footnotesize
	Nov}}}
    {CEB, University of Minho}
    {Researcher}
    {Computação Colectiva e Controlo em Sistemas\\
	Bioquímicos Complexos}
  \entry
    {\parbox[t][][t]{1.8cm}{2011 {\footnotesize Apr to}\\2012 {\footnotesize Sep}}}
    {CEHUM, University of Minho}
    {Researcher}
    {Project Per-Fide: Portuguese in parallel with six languages:\\
	Español, Russian, Français, Italiano, Deutsch, English\\
	{\footnotesize
	\url{http://per-fide.di.uminho.pt/}
	}}
  \entry
    {2009 {\footnotesize Jul-Sep}}
    {Sapo Summerbits'09}
    {Research grant}
    {Open source software development\\
	\footnotesize{
	Development of natural language processing tools to help in general\\
	orthography migration challenges, motivated by the 1990 Portuguese\\
	Language Orthographic Agreement.\\
	\url{http://softwarelivre.sapo.pt/projects/bigorna/}}}
%  \entry
%    {06–08 2008}
%    {ISCPIF/CNRS, Paris}
%    {Research Internship.}
%    {\emph{Diffusion in the Blogosphere. Happy Flu.}}
%  \entry
%    {06–08 2007}
%    {LIP6/CNRS, Paris}
%    {Research Internship.}
%    {\emph{Kernels in real world networks.}}
%  \entry
%    {07–08 2005}
%    {\href{http://www.kelkoo.com}{Kelkoo.com}}
%    {Summer job.}
%    {\emph{Creation of a keyword generator for Google Adwords.}}
%  \entry
%    {07–08 2004}
%    {\href{http://www.monsieurprix.com}{MonsieurPrix.com}}
%    {Summer job.}
%    {\emph{Development of an e-commerce product indexation spider.}}
\end{entrylist}


\section{technical skills}
\begin{entrylist}
	\entry
		{}
		{Perl}
		{Proficient}
		{Including Moose, Catalyst, Dancer, DBIx::Class and\\
		other Modern::Perl concepts.}
	\entry
		{}
		{Web development}
		{Proficient}
		{Mostly using Perl frameworks, but also Bootstrap\\
		from Twitter, CSS, Javascript.}
	\entry
		{}
		{Ontologies and structured documents}
		{Experienced}
		{XML, XQuery, XPath, JSON and ontology formats: OWL,\\
		OBO, RDF.}
\end{entrylist}
\begin{entrylist}
	\entry
		{}
		{Systems administration, programming tools and methods}
		{Experienced}
		{Bash, Subversion, Git, \LaTeX, Apache, Agile.}
	\entry
		{}
		{Other programming languages}
		{}
		{SQL, C, Java, PHP, UML, Haskell, Ruby, Pascal.}
		% Meter aqui coisas de text mining
\end{entrylist}

\newpage
\section{education}

\begin{entrylist}
  \entry
    {2011--2012}
    {MSc in Bioinformatics}
    {University of Minho, Portugal}
    {\emph{Mining biological parameters from literature: an\\
	application to environmental decision support systems}\\
	{\footnotesize{Development of a modular and extensible framework
	for building\\
	text mining applications for scientific literature, focused on
	numerical\\ 
	parameters and including a web interface.}}\\
    120 ECTS, final grade 17/20
	}
  \entry
    {2009--2011}
    {MSc in Natural Language Processing}
    {University of Minho, Portugal}
    {\emph{Contributions for building a Corpora-Flow System}\\
	{\footnotesize{Development of a system to support automatic handling of corpora\\
	(cleaning, validation, format conversion), document alignment and\\
	calculation alignability metrics of several types of documents.}}\\
    120 ECTS, final grade 17/20
	}
  \entry
    {2006--2009}
    {BSc in Software Engineering}
    {University of Minho, Portugal}
    {180 ECTS, final grade 15/20}
\end{entrylist}


\section{training}
\begin{entrylist}
	\entry
		{2011 {\footnotesize Jul}}
		{Bang and Olufsen Summer School}
		{Struer, Denmark}
		{Conceptual Design and Development of Innovative Products}
	\entry
		{2010/11}
		{IdeaLab -- Business Ideas Lab}
		{TecMinho}
		{\vspace{-0.8cm}}
	\entry
		{2010 {\footnotesize Jul}}
		{Luso-Brazilian Evolutionary Computing School}
		{University of Minho, Guimarães}
		{\vspace{-0.8cm}}
	\entry
		{2010 {\footnotesize Jun}}
		{Catalyst 5.80}
		{Portuguese Perl Workshop'10}
		{\vspace{-0.8cm}}
	\entry
		{2005 {\footnotesize Set}}
		{Physics Summer School}
		{Faculty of Science, University of Porto}
		{\vspace{-0.8cm}}
	\entry
		{2005 {\footnotesize Jul}}
		{German Open Course}
		{Faculty of Arts, Univerity of Porto}
		{\vspace{-0.8cm}}
\end{entrylist}


\section{other activities}
\subsection{cesium {\normalfont\small University of Minho's Software Engineering Student Center}}{}
\begin{entrylist}
  \entry
    {2009/10}
    {Vice-president}
    {\href{http://cesium.di.uminho.pt}{http://cesium.di.uminho.pt}}
    {\vspace{-.8cm}}
  \entry
    {2010/11}
    {Director at CAOS}
    {\href{http://caos.di.uminho.pt}{http://caos.di.uminho.pt}}
    {CeSIUM's Open Source Support Center}
\end{entrylist}

\subsection{programming challenges}{}
\begin{entrylist}
  \entry
    {2012}
    {APPP Perl Code Sprint}
	{\href{http://perl.pt}{http://perl.pt}}
    {3$^{\footnotesize\textrm{rd}}$~qualified in the Perl programming competition\\organized by the Portuguese Perl Mongers group.}
  \entry
    {2009}
    {Microsoft Imagine Cup'09}
	{Lisbon, Portugal}
    {5$^{\footnotesize\textrm{th}}$~classified in the national finals.}
  \entry
    {2008}
    {ACM South Western European Regional Contests}
	{Nuremberg, Germany}
    {International programming competition.}
  \entry
    {2007-2009}
    {Inter-University Programming Marathon}
	{IST'07, UC'08, ESTGA'09}
    {National ACM-style programming competition.}
\end{entrylist}

\subsection{sports}{}
\begin{entrylist}
  \entry
    {{\footnotesize since} 1999}
    {Judo}
    {INATEL; University of Minho}
    %{4$^{\footnotesize\textrm{th}}$~Kyu}
    {\vspace{-.8cm}}
  \entry
    {2010-2012}
    {National Judo University Championships}
    {Aveiro'10, Lisbon'11, Aveiro '12}
    {\vspace{-.8cm}}
  \entry
    {2007/08}
    {Basketball}
    {University of Minho}
    {\vspace{-.8cm}}
  \entry
    {1998-2000}
    {Artistic gymnastics}
    {INATEL}
    {}
\end{entrylist}

\newpage
\section{publications}
\printbibsection{article}{articles in peer-reviewed journal}

\begin{refsection}
  \nocite{*}
  %\printbibliography[type=inproceedings, title={international peer-reviewed conferences/proceedings}, notkeyword={france}, heading=subbibliography]
  \printbibliography[type=inproceedings, title={peer-reviewed conferences/proceedings}, heading=subbibliography]
\end{refsection}

\printbibsection{thesis}{dissertations}

\section{presentations}
\begin{refsection}
  \nocite{*}
  \printbibliography[type=misc, title={presentations}, heading=none]
\end{refsection}

\end{document}
